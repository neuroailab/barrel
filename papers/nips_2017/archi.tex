In this section, we will show several families of DNNs trained on the dataset. Different families of DNNs are based on different hypotheses about the computations performed by neurons. 
For example, the model with spatial-temporal convolution structure is based on our assumption about neurons integrating spatial-temporal information. 
As the temporal dimension of our input data is fixed across different trials, we could use feed-forward convolutional neural networks (CNNs) as well as recurrent neural networks (RNNs).
We will compare different families of networks through their performances while controlling the number of parameters in the model.

\subsection{Spatial-temporal integrating networks}

In this family of networks, we use CNNs where convolution is done simultaneously on temporal dimension and spatial dimension, which means that responses from different whiskers across a specific window of time will be combined together in neurons of every layer, while neurons at higher layer will have larger receptive field at both dimensions.
We began with 2D-convolution on temporal and spatial dimension, where the spatial dimension is the indexing of the whiskers through going across the columns and then the rows on the $5\times7$ grid. 
We also tried expanding the spatial dimension into two dimensions of $5\times7$ and then doing a 3D-convolution instead, the results will be shown later in this section. 

The general structure of the networks includes several convolution layers followed by several fully connected layers. Each categorization is done towards a combination of top, middle, and bottom swipe of the same object under fixed setting. 
The same network will be applied to three swipes after which the outputs will be concatenated to give the final categorization label.
For example, one structure in this family consists of 5 convolutional layers and 2 fully connected layers for each swipe with one additional fully connected layer for combining three swipes. 
The convolutional layers are named from \textit{conv1} to \textit{conv5}.
And the filter size of \textit{conv1} is $9\times3$ and that of other convolutional layers is $3\times3$, where the first number is for temporal dimension and the second number is for spatial dimension. 
The strides of all convolutional layers for both dimensions are 1. The number of filters are 96, 256, 384, 384, and 256 respectively.
There are max-pooling layers after \textit{conv1}, \textit{conv2}, and \textit{conv5}, called \textit{pool1}, \textit{pool2}, and \textit{pool5}. The filter size of \textit{pool1} is $3\times1$ and the stride of \textit{pool1} is the same. 
For \textit{pool2} and \textit{pool5}, the filter size is $3\times3$ while the stride is $2\times2$.
After \textit{pool5}, the output is transformed by two fully connected layers \textit{fc6} and \textit{fc7}. Layer \textit{fc6} has an output shape of 4096 and layer \textit{fc7} has an output shape of 1024.
Finally the three swipes are concatenated together and an additional fully connected layer \textit{fc_add} is used to predict the category by giving an output with shape of 117.

The network is trained using cross-entropy loss function using Adagrad algorithm \cite{duchi2011adaptive}. 
The learning rate remained at 

\subsection{Temporal integrating before spatial integrating}

\subsection{Spatial integrating before temporal integrating}

\subsection{Recurrent neural networks}
